%%%%%%%%%%%%%%%%%%%%%%%%%%%%%%%%%%%%%%%%%%%%%%%%%%%%%%%%%%%%%%%%%%%%%%%%%%%%%%%
% CAPÍTULO 1
\chapter{INTRODUÇÃO}  

\notain{Exemplo}{Este é um exemplo de comentário visual inline.}

O presente documento \'e um exemplo de uso do estilo de formata\c{c}\~ao \LaTeX\ elaborado para atender \`as Normas para Elabora\c{c}\~ao de Trabalhos Acad\^emicos da UTFPR. 
\notaOri{Exemplo de comentário visual lateral do Orientador.}
O estilo de formata\c{c}\~ao {\ttfamily normas-utf-tex.cls} tem por base o pacote \textsc{abn}\TeX~-- cuja leitura da documenta\c{c}\~ao \cite{abnTeX2009} \'e fortemente sugerida~-- e o estilo de formata\c{c}\~ao \LaTeX\ da UFPR.

\begin{sloppypar} % Esta anotação força para que o parágrafo não exceda a margem e respeite o justificado
Para melhor entendimento do uso do estilo de formata\c{c}\~ao {\ttfamily normas-utf-tex.cls}, aconselha-se que o potencial usu\'ario analise os comandos existentes no arquivo \TeX\ ({\ttfamily modelo\_*.tex}) e os resultados obtidos no arquivo PDF ({\ttfamily modelo\_*.pdf}) depois do processamento pelo software \LaTeX\ + \textsc{Bib}\TeX~\cite{LaTeX2009,BibTeX2009}. 
\notaAlu{Exemplo de comentário visual lateral do Aluno.}
Recomenda-se a consulta ao material de refer\^encia do software para a sua correta utiliza\c{c}\~ao~\cite{Lamport1986,Buerger1989,Kopka2003,Mittelbach2004}.
\end{sloppypar}
	
\section{Motiva\c{c}\~ao}

Uma das principais vantagens do uso do estilo de formata\c{c}\~ao {\ttfamily normas-utf-tex.cls} para \LaTeX\ \'e a formata\c{c}\~ao \textit{autom\'atica} dos elementos que comp\~oem um documento acad\^emico, tais como capa, folha de rosto, dedicat\'oria, agradecimentos, ep\'igrafe, resumo, abstract, listas de figuras, tabelas, siglas e s\'imbolos, sum\'ario, cap\'itulos, refer\^encias, etc. Outras grandes vantagens do uso do \LaTeX\ para formata\c{c}\~ao de documentos acad\^emicos dizem respeito \`a facilidade de gerenciamento de refer\^encias cruzadas e bibliogr\'aficas, al\'em da formata\c{c}\~ao~-- inclusive de equa\c{c}\~oes  matem\'aticas~-- correta e esteticamente perfeita.

\section{Objetivos}

\subsection{Objetivo Geral}

Prover um modelo de formata\c{c}\~ao \LaTeX\ que atenda \`as Normas para Elabora\c{c}\~ao de Trabalhos Acad\^emicos da UTFPR~\cite{UTFPR2008} e \`as Normas de Apresenta\c{c}\~ao de Trabalhos Acad\^emicos do DAELN~\cite{DAELN2006}.

\subsection{Objetivos Espec\'ificos}

\begin{itemize}
	\item Obter documentos acad\^emicos automaticamente formatados com corre\c{c}\~ao e perfei\c{c}\~ao est\'etica.
	\item Desonerar autores da tediosa tarefa de formatar documentos acad\^emicos, permitindo sua concentra\c{c}\~ao no conte\'udo do mesmo.
	\item Desonerar orientadores e examinadores da tediosa tarefa de conferir a formata\c{c}\~ao de documentos acad\^emicos, permitindo sua concentra\c{c}\~ao no conte\'udo do mesmo.
\end{itemize}

\section{Organização do Texto}

